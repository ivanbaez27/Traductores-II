%************AVISO*************************************
%***Hay que definir las variables siguientes:**********
%***\elalumno linea 19
%***\tituloPractica linea 24
%***\fecha linea 22  para cambiar fecha

%***
\documentclass[11pt]{article}
\usepackage[utf8]{inputenc}
\usepackage[spanish]{babel}
\usepackage{amsmath,cancel}
\usepackage{amsfonts}
\usepackage{amssymb}
\usepackage{graphicx}
\usepackage{float}
\usepackage{cite}

%Cambiar nombre del alumno
\newcommand{\elalumno}{Nombre del alumno}
\newcommand{\director}{Erasmo Gabriel Martínez Soltero}
%Cambiar fecha 28 agosto 2017
\newcommand{\fecha}{28 agosto 2017}
%Cambiar Titulo de la practica
\newcommand{\tituloPractica}{Título de la práctica}
%%%%%%%%%%%%%%%%%%%%%%%%%%%%%%%%%%%%%%%%%%%%%%%%%%%%%%%%%%%%%%%%%
%%% ----> Esto dejarlo igual
\addtolength{\textheight}{90pt}
\begin{document}
\begin{center}
%\vspace*{-3.5cm}
  {\Large \textbf{UNIVERSIDAD DE GUADALAJARA}}\\
  \begin{figure} [H]
	\begin{center}
  		\includegraphics[width=0.3\textwidth]{udg}
	\end{center}
  \end{figure}
  {\large \textbf{CENTRO UNIVERSITARIO DE CIENCIAS EXACTAS E INGENIERÍAS}}\\
  \vspace{0.25cm}
{\large \textbf{Inteligencia Artificial}}\\
\vspace{0.25cm}
\vspace{0.5cm}
{\large Reporte de práctica} \vspace{1cm} \\

\begin{tabular}{p{4cm}p{8cm}}
Nombre del alumno: & \elalumno \\
Profesor: & \director \\
T\'\i tulo de la práctica: &
``{\tituloPractica}'' \\
Fecha: & \fecha
\end{tabular}
\end{center}
%%%%%%%%%%%%%%%%%%%%%%%%%%%%%%%%%%%%%%%%%%%%%%%%%%%%%%%%%%%%%%%%%
%%%%%%%%%%%%%%%%%%%%%%%%%%%%%%%%%%%%%%%%%%%%%%%%%%%%%%%%%%%%%%%%%
%% Inician con el contenido de la información 
\section*{Introducción}

En esta sección se hace una breve descripción del problema a resolver.

\section*{Metodología}

Se describe el método para resolver la problemática (el algoritmo a implementar) haciendo uso de herramientas como diagramas de flujo, pseudocódigo, etc.

\section*{Resultados}

En esta sección se muestran los resultados obtenidos así como los datos de entrada utilizados para la ejecución del programa; incluya también capturas de pantalla.

\section*{Conclusiones}

Se redactan las conclusiones sobre el algoritmo implementado, mencionando ventajas y desventajas. También se pueden hacer observaciones personales sobre la metodología o el problema que se resolvió.

\section*{Referencias}
Se enumeran las fuentes bibliográficas utilizadas para la realización de la práctica.

\end{document}
